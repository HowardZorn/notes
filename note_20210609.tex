% !TEX TS-program = xelatex
% !TEX encoding = UTF-8

\documentclass[a4paper,punct,fancyhdr]{ctexart}

\usepackage{amsmath, amssymb, amsfonts}
\defaultfontfeatures{Mapping=tex-text} % to support TeX conventions like ``---''
\usepackage{xunicode} % Unicode support for LaTeX character names (accents, European chars, etc)
\usepackage{xltxtra} % Extra customizations for XeLaTeX\usepackage{geometry} % See geometry.pdf to learn the layout options. There are lots.
\usepackage{geometry} % See geometry.pdf to learn the layout options. There are lots.
\geometry{a4paper} % or letterpaper (US) or a5paper or....
%\usepackage[parfill]{parskip} % Activate to begin paragraphs with an empty line rather than an indent
\usepackage{graphicx} % support the \includegraphics command and options
\usepackage{hyperref}

\title{Multi-Task Learning Using Uncertainty to Weigh Losses for Scene Geometry and Semantics\\笔记}
\author{Fw[a]rd}
\date{2021年6月9日}

\begin{document}
\maketitle

\section{因缘}
因为某个项目的原因,需要考虑多个loss的平衡问题。在\href{https://www.zhihu.com/question/375794498}{知乎}用户的推荐下阅读了此文。

\section{内容}
\subsection{背景}
\subsubsection{Multi-Task Learning}
把多个相关的任务放在一起学习,同时学习多个任务。多个任务之间共享一些因素,它们可以在学习过程中,共享它们所学到的信息,这是单任务学习所不具备的。相关联的多任务学习比单任务学习能去的更好的泛化效果。广义的讲,只要有多个loss就算multi-task learning。

\subsection{贡献}
\begin{enumerate}
    \item 一种全新的有道理的multi-task loss,它的效果能比采用固定权重相加的多个loss好很多。
    \item 一个包含了语义分割、实例分割、深度回归的统一的架构
\end{enumerate}

\end{document}