% !TEX TS-program = xelatex
% !TEX encoding = UTF-8

\documentclass[a4paper,punct]{ctexart}
\usepackage{note}

\title{Multi-Task Learning 笔记}
\author{Fw[a]rd\thanks{这篇文章的内容以GPLv2协议释出}}
\date{2021年6月9日}

\begin{document}
\maketitle

\section{因缘}
因为某个项目的原因,需要考虑多个loss的平衡问题。在\href{https://www.zhihu.com/question/375794498}{知乎}用户的推荐下阅读了:

\begin{itemize}
    \item Multi-Task Learning Using Uncertainty to Weigh Losses for Scene Geometry and Semantics\cite{Kendall18Uncertainty}
    \item GradNorm: Gradient Normalization for Adaptive Loss Balancing in Deep Multitask Networks\cite{Chen18GradNorm}
    \item Multi-Task Learning as Multi-Objective Optimization\cite{Sener18Pareto}
\end{itemize}

这几篇文章各有千秋,方法论不尽相同。

\section{内容}
\subsection{背景}
\subsubsection{Multi-Task Learning}
即把多个相关的任务放在一起学习。多个任务之间共享一些参数,它们可以在学习过程中共享它们所学到的信息,这是单任务学习所不具备的。相关联的多任务学习比单任务学习能去的更好的泛化效果。更加广义地,只要有多个loss就算multi-task learning。

\subsection{方法}

之前多重任务学习问题大多采用多个loss的加权和作为整个任务的loss:
\begin{equation}
    L_\mathrm{total} = \sum_i w_iL_i.
\end{equation}
这些权值$w$通常是不变的并且需要手动调整,这就让获得最优权值的过程很麻烦。

\subsubsection{\citefullauthor{Kendall18Uncertainty}}

\citet{Kendall18Uncertainty}提出,多重任务学习中每个任务最优的权重由其测量尺度——更详细地,由其噪声的大小决定。他们将\textbf{同方差不确定性}理解为与任务相关的权值,基于同方差不确定性的高斯极大似然推导出了一套多重任务loss。

假设$f^W(x)$是神经网络$f^W(\cdot)$的输出,$W$是神经网络的权值,$x$是输入。那么对于回归问题和分类问题的概率模型可建模为:
\begin{equation}
    \label{eq:output-likelihood}
    \begin{aligned}
        p(y|f^W(x)) &= \mathcal{N}(f^W(x), \sigma^2) = {\frac {1}{\sigma {\sqrt {2\pi }}}}\;e^{-{\frac {\left(y-f^W(x) \right)^{2}}{2\sigma ^{2}}}}, &\textrm{a.回归问题},\\
        p(y|f^W(x)) &= \mathrm{Softmax}(f^W(x)) \xlongequal{\textrm{第i个元素}} \frac{e^{f^W_i(x)}}{\sum_{i} e^{f^W_i(x)}}, &\textrm{b.分类问题},
    \end{aligned}
\end{equation}
其中,$\mathcal{N}$是高斯分布,$\sigma$是代表观测噪声水平的标量。

对式~\ref{eq:output-likelihood}.a~做对数,可得其对数似然正比于:
\begin{equation}
    \label{eq:回归问题公式}
    \begin{aligned}
        \log p(y|f^W(x)) &\propto - \frac{1}{2\sigma^2}\|y-f^W(x)\|^2 - \log \sigma \\
        &= -\frac{1}{2\sigma^2}\mathcal{L_\mathrm{MSE}}(W) - \log \sigma.
    \end{aligned}
\end{equation}

对式~\ref{eq:output-likelihood}.b~进行缩放,然后对$p(y|f^W(x)) = \mathrm{Softmax}(\frac{1}{\sigma^2} f^W(x))$\footnote{可以认为是\href{https://zh.wikipedia.org/wiki/玻尔兹曼分布}{玻尔兹曼分布}}做对数,可得:
\begin{equation}
    \log p(y|f^W(x)) \xlongequal{\textrm{第i个元素}} \frac{1}{\sigma^2} f^W_i(x) - \log \sum_e \exp\left( \frac{1}{\sigma^2} f^W_e(x)\right).
\end{equation}

然后可推对于分类问题不进行缩放的对数似然函数(不要问我为什么,反复看了论文数次没有看懂……):% TODO 搞懂公式推导过程……
\begin{equation}
    \label{eq:分类问题公式}
    \log p(y|f^W(x)) \approx -\frac{1}{\sigma^2}\mathcal{L_\mathrm{Cross Entropy}}(W) - \log \sigma.
\end{equation}

对于\emph{多}个任务的多个输出,其极大似然及对数似然可以表示为:
\begin{equation}
    \begin{aligned}
        p(y_1, \ldots, y_k|f^W(x)) &= p(y_1 | f^W(x)) \cdots p(y_k | f^W(x)), \\
        &\downarrow \\
        \log p(y_1, \ldots, y_k|f^W(x)) &= \log p(y_1 | f^W(x)) + \cdots + \log p(y_k | f^W(x)).
    \end{aligned}
\end{equation}

因为loss追求最小化,所以给多个任务的对数似然添加负号之后就是我们要的多重任务loss函数$\mathcal{L}(\cdot)$,可以使用式~\ref{eq:回归问题公式}~和式~\ref{eq:分类问题公式}的结果对其中的单个任务的输出的对数似然进行替换。
\begin{equation}
    \mathcal{L}(W) = - \log p(y_1, \ldots, y_k|f^W(x)) = - (\log p(y_1 | f^W(x)) + \cdots + \log p(y_k | f^W(x))).
\end{equation}
在这个多重任务loss中,代表噪声的$\sigma$将作为可学习的参数。因为在$\log(\cdot)$下$\sigma$可能取到非法的值,所以我们重新定义一个可学习参数$s=\log \sigma^2$代替式子中的$\sigma$。

\paragraph{例} 假设有一个多重任务,其中一个是回归问题($\mathcal{L}_1$是MSE),一个是分类问题($\mathcal{L}_2$是交叉熵)。那么它总的loss是:
\begin{equation}
    \begin{aligned}
        \mathcal{L}(W) &= - \log p(y_1|f^W(x)) - \log p(y_2|f^W(x)) \\
         &= \frac{1}{2{\sigma_1}^2} \mathcal{L}_1(W) + \frac{1}{{\sigma_2}^2} \mathcal{L}_2(W) + \log \sigma_1\sigma_2 \\
         &\xlongequal{s_i =\log {\sigma_i}^2} \frac{1}{2{e}^{s_1}} \mathcal{L}_1(W) + \frac{1}{{e}^{s_2}} \mathcal{L}_2(W) + \frac{s_1 + s_2}{2}.
    \end{aligned}
\end{equation}

\subsubsection{\citefullauthor{Chen18GradNorm}}

\citet{Chen18GradNorm}提出,

\subsubsection{\citefullauthor{Sener18Pareto}}

\citet{Sener18Pareto}提出,

\subsection{实现}

\subsubsection{\citefullauthor{Kendall18Uncertainty}}

官方实现是一个空的仓库,很有意思的是把issue给关了(GitHub: \href{https://github.com/dyz-zju/multitaskvision}{dyz-zju/multitaskvision})。有非官方实现(GitHub: \href{https://github.com/ranandalon/mtl}{ranandalon/mtl}, \href{https://github.com/Mikoto10032/AutomaticWeightedLoss}{Mikoto\-10032/Automatic\-Weighted\-Loss})。

\begin{minted}{python}
import torch
import torch.nn as nn

class AutomaticWeightedLoss(nn.Module):
    """automatically weighted multi-task loss
    Params:
        num: int,the number of loss
        x: multi-task loss
    Examples:
        loss1=1
        loss2=2
        awl = AutomaticWeightedLoss(2)
        loss_sum = awl(loss1, loss2)
    """
    def __init__(self, num=2):
        super(AutomaticWeightedLoss, self).__init__()
        params = torch.ones(num, requires_grad=True)
        self.params = torch.nn.Parameter(params)

    def forward(self, *x):
        loss_sum = 0
        for i, loss in enumerate(x):
            loss_sum += 0.5 / (self.params[i] ** 2) * loss + \
                torch.log(1 + self.params[i] ** 2)
        return loss_sum
\end{minted}


\section{结果与观感}



\bibliography{ref/note_20210609}

\end{document}