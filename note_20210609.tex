% !TEX TS-program = xelatex
% !TEX encoding = UTF-8

\documentclass[a4paper,punct]{ctexart}

\usepackage{amsmath, amssymb, amsfonts}
\defaultfontfeatures{Mapping=tex-text} % to support TeX conventions like ``---''
\usepackage{xunicode} % Unicode support for LaTeX character names (accents, European chars, etc)
\usepackage{xltxtra} % Extra customizations for XeLaTeX\usepackage{geometry} % See geometry.pdf to learn the layout options. There are lots.
\usepackage[margin={2.54cm,3.18cm}]{geometry} % See geometry.pdf to learn the layout options. There are lots.
\geometry{a4paper} % or letterpaper (US) or a5paper or....
%\usepackage[parfill]{parskip} % Activate to begin paragraphs with an empty line rather than an indent
\usepackage{graphicx} % support the \includegraphics command and options
\usepackage{hyperref}
\usepackage{minted}
\usepackage[numbers,super,square]{natbib}
\bibliographystyle{unsrtnat}

\title{Multi-Task Learning 笔记}
\author{Fw[a]rd\thanks{这篇文章的内容以GPLv2协议释出}}
\date{2021年6月9日}

\begin{document}
\maketitle

\section{因缘}
因为某个项目的原因,需要考虑多个loss的平衡问题。在\href{https://www.zhihu.com/question/375794498}{知乎}用户的推荐下阅读了:

\begin{itemize}
    \item Multi-Task Learning Using Uncertainty to Weigh Losses for Scene Geometry and Semantics\cite{Kendall18Uncertainty}
    \item GradNorm: Gradient Normalization for AdaptiveLoss Balancing in Deep Multitask Networks\cite{Chen18GradNorm}
    \item Multi-Task Learning as Multi-Objective Optimization\cite{Sener18Pareto}
\end{itemize}

这几篇文章各有千秋,方法论不尽相同。

\section{内容}
\subsection{背景}
\subsubsection{Multi-Task Learning}
把多个相关的任务放在一起学习,同时学习多个任务。多个任务之间共享一些因素,它们可以在学习过程中,共享它们所学到的信息,这是单任务学习所不具备的。相关联的多任务学习比单任务学习能去的更好的泛化效果。更加广义地,只要有多个loss就算multi-task learning。

\subsection{方法}

之前多重任务学习问题大多采用多个loss的加权和作为整个任务的loss:

\begin{equation}
    L_\mathrm{total} = \sum_i w_iL_i
\end{equation}

这些权值$w$通常是不变的并且需要手动调整,这就让获得最优权值的过程很麻烦。

\subsubsection{\citefullauthor{Kendall18Uncertainty}}

\citet{Kendall18Uncertainty}提出,多重任务学习中每个任务最优的权重由其测量尺度——更详细地,由其噪声的大小决定。\citet{Kendall18Uncertainty}将\textbf{同方差不确定性}理解为与任务相关的权值。他们基于同方差不确定性的高斯极大似然推导出了一套多重任务loss。

假设$f^W(x)$是神经网络$f^W(\cdot)$的输出,$W$是神经网络的权值,$x$是输入。那么对于回归问题和分类问题的概率模型可建模为:

\begin{equation}
    \begin{aligned}
        p(y|f^W(x)) &= \mathcal{N}(f^W(x), \sigma^2),\textrm{回归问题},\\
        p(y|f^W(x)) &= \mathrm{Softmax}(f^W(x)),\textrm{分类问题}.
    \end{aligned}
\end{equation}

其中,$\mathcal{N}$是高斯分布,$\sigma$是代表观测噪声水平的标量。

\subsection{实现}

\subsubsection{\citefullauthor{Kendall18Uncertainty}}

似乎没有官方实现。有非官方实现\href{https://github.com/ranandalon/mtl}{GitHub: ranandalon/mtl}。

% \begin{minted}{python}
% class PositionalEncoding(nn.Module):
%     '''
%     Positional Encoding
%     '''
%     def __init__(self):
%         super(PositionalEncoding, self).__init__()

%     def forward(self, shape, device):
%         '''
%         shape: Size(batch_size, P, N, D)
%         device: tensor's device type
%         return: [batch_size, P, N, D]
%         '''
%         batch_size, P, N, D = shape
%         return torch_utils.positional_encoding_traffic(batch_size, P, N, D).to(device)
% \end{minted}

\bibliography{ref/note_20210609}

\end{document}